\documentclass[a4paper,12pt]{article}

\usepackage[latin1]{inputenc}
\usepackage[T1]{fontenc}
\usepackage[francais]{babel}
\usepackage{graphicx}
\usepackage{color}
\definecolor{grey}{rgb}{0.9,0.9,0.9}
\definecolor{teal}{rgb}{0.0,0.5,0.5}
\definecolor{violet}{rgb}{0.5,0,0.5}
\usepackage[margin=2.5cm]{geometry}\usepackage{listings}
\usepackage{listingsutf8}
\lstloadlanguages{caml}
\lstdefinestyle{listing}{
  language=[Objective]Caml,
  captionpos=t,
  inputencoding=utf8/latin1,
  extendedchars=true,
  numbers=left,
  numberstyle=\tiny,
  numbersep=5pt,
  breaklines=true,
  breakatwhitespace=true,
  showspaces=false,
  showstringspaces=false,
  showtabs=false,
  tabsize=2,
  basicstyle=\footnotesize\ttfamily,
  backgroundcolor=\color{grey},
  keywordstyle=\color{blue}\bfseries,
  commentstyle=\color{teal},
  identifierstyle=\color{black},
  stringstyle=\color{red},
  numberstyle=\color{violet},
}
\lstset{style=listing}

\title{TP1 - Analyse lexicale et crible}
\author{\textsc{Rapha�l Baron} - \textsc{Paul Chaignon}}
\date{\today}

\begin{document}

\maketitle

\section{Analyseur}

\lstinputlisting[caption=lexer.ml]{ulex.ml}

\lstinputlisting[caption=lexer.mll]{lexer.mll}
\vspace{1cm}

\section{Tests}

\lstinputlisting[caption=tests.ml]{tests.ml}
\vspace{1cm}

\section{Questions}

\subsection{Question 1.1}
Avoir un crible s�par� permet de simplifier l'automate. Tout d'abord, le lex�me va �tre reconnu par l'automate (valide ou non), puis gr�ce au crible on pourrra lui attribuer l'unit� lexicale qui correspond. Au final, la maintenace de code sera plus ais�e.

\subsection{Question 1.2}
Oui, c'est parfaitement possible. Cela n'affectera pas l'AFD d'OCamllex, cependant, le crible sera lui modifi� (afin d'associer correctement les unit�s lexicales et/ou).

\subsection{Question 1.3}
On peut distinguer 2 int�r�ts principaux:
	- On peut facilement effectuer un filtrage (objectif de l'analyse syntaxique), gr�ce aux "match .. with .."
	- L'ajout de nouveaux �l�ments reconnus est tr�s simple : il suffit d'ajouter une ligne dans la d�finitions des types reconnus.
Enfin, m�me si c'est plus anecdotique, cela permet d'avoir un code plus propre, et rapidement compr�hensible.

\subsection{Question 1.4}
Gr�ce au scanner incr�mental, on peut plus facilement voir o� se trouver l'erreur, lorsqu'il y en a une. En effet, un scanner traditionnel permet simplemetn de savoir si l'expression est ou non reconnue, mais pas de situer l'erreur. A l'inverse, en incr�mentant on sait exactement quel token pose probl�me.

\subsection{Question 1.5}
On consid�re ident comme un terminal, car ici on n'applique qu'une analyse lexicale, et non syntaxique. Un ident est un lex�me, et est donc terminal.

\subsection{Question 1.6}
Il suffit de cr�er trois nouvelles unit�s lexicales : UL\_SI, UL\_ALORS et UL\_SINON, et de les sp�cifier dans le crible.

\end{document}