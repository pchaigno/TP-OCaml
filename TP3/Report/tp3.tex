\documentclass[a4paper,12pt]{article}

\usepackage[latin1]{inputenc}
\usepackage[T1]{fontenc}
\usepackage[francais]{babel}
\usepackage{graphicx}
\usepackage{color}
\definecolor{grey}{rgb}{0.9,0.9,0.9}
\definecolor{teal}{rgb}{0.0,0.5,0.5}
\definecolor{violet}{rgb}{0.5,0,0.5}
\usepackage{array}
\usepackage{amsmath}
\usepackage{amsfonts}
\usepackage{amssymb}
\usepackage[normalem]{ulem}
\usepackage{amsthm}
\usepackage[margin=2.5cm]{geometry}
\usepackage{listings}
\usepackage{listingsutf8}
\lstloadlanguages{caml}
\lstdefinestyle{listing}{
  language=[Objective]Caml,
  captionpos=t,
  inputencoding=utf8/latin1,
  extendedchars=true,
  numbers=left,
  numberstyle=\tiny,
  numbersep=5pt,
  breaklines=true,
  breakatwhitespace=true,
  showspaces=false,
  showstringspaces=false,
  showtabs=false,
  tabsize=2,
  basicstyle=\footnotesize\ttfamily,
  backgroundcolor=\color{grey},
  keywordstyle=\color{blue}\bfseries,
  commentstyle=\color{teal},
  identifierstyle=\color{black},
  stringstyle=\color{red},
  numberstyle=\color{violet},
}
\lstset{style=listing}
\newcommand{\chevron}[1]{<\! \! #1 \! \!>}

\title{TP3 - Analyse syntaxique ascendante, ocamllex et ocamlyacc}
\author{\textsc{Rapha�l Baron} - \textsc{Paul Chaignon}}
\date{19 novembre 2013}

\begin{document}

	\maketitle

	\section{Pr�paration du TP}
	
		\subsection{Table SLR}

			\begin{table}[h]
				\begin{center} 
					\begin{tabular}{|c|c|c|c|c|c|c|c|c|c|c|}
						\hline
						& Ident & $\leftarrow$ & $+$ & $<$ & and & ( & ) & \$ & Inst & E \\
						\hline
						I0 & S2 &  &  &  &  &  &  &  & 1 &  \\
						\hline  
						I1 &  &  &  &  &  &  &  &  Acc &  &  \\
						\hline  
						I2 &  & S3 &  &  &  &  &  &  &  &  \\
						\hline  
						I3 & S6 &  &  &  &  & S5 &  &  &  & 4 \\
						\hline
						I4 &  &  & S7/\sout{R1} & S8/\sout{R1} & S9/\sout{R1} &  & R1 & R1 &  &  \\
						\hline
						I5 &  &  &  &  &  &  &  &  &  &  10\\
						\hline
						I6 &  &  & R0 & R0 & R0 &  & R0 & R0 &  &  \\
						\hline
						I7 & S6 &  &  &  &  & S5 &  &  &  & 11 \\
						\hline
						I8 & S6 &  &  &  &  & S5 &  &  &  & 12 \\
						\hline
						I9 & S6 &  &  &  &  & S5 &  &  &  & 13 \\
						\hline
						I10 &  &  &  &  &  &  & S14 &  &  &  \\
						\hline
						I11 &  &  & \sout{S7}/R1 & \sout{S8}/R1 & \sout{S9}/R1 &  & R1 & R1 &  &  \\
						\hline
						I12 &  &  & S7/\sout{R2} & \sout{S8}/R2 & \sout{S9}/R2 &  & R2 & R2 &  &  \\
						\hline
						I13 &  & & S7/\sout{R3} & S8/\sout{R3} & \sout{S9}/R3 &  & R3 & R3 &  &  \\
						\hline
						I14 &  & & R4 & R4 & R4 &  & R4 & R4 &  &  \\
						\hline
					\end{tabular}
					\caption{Table SLR avec conflits}
				\end{center}
			\end{table}

		\subsection{R�solution des conflits}
			Pour r�soudre les confilts, on utilise le fait que le \textit{And} a une priorit� moins importante que le $\textit{<}$ qui a aussi une priorit� moins importante que le $\textit{+}$.

			

	\section{Questions}
	
		\subsection*{Question 1}
			L'utilisation d'un crible avec ocamlex permet de r�duire le nombre d'�tats (et de transitions).

		\subsection*{Question 2}
			Il n'y a pas besoin de lever les ambiguit�s pour une grammaire LR mais l'analyseur ascendant pour la grammaire LR est plus compliqu�.

		\subsection*{Question 3}
			Si la colonne d'un terminal est vide, cela signifie que ce terminal ne sera jamais rencontr�.


	\section{Codes source}
	
		\lstinputlisting[caption=grammar.mly]{../grammar.mly}
		\lstinputlisting[caption=lexer.mll]{../lexer.mll}
		\lstinputlisting[caption=arbre.ml]{../definitions.ml}

\end{document}